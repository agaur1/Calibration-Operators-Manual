
Control of the calibration manipulator and the calibration laser is
done through the \textbf{calibration} computer running the \textbf{manip}
program which is a C++ program running on Linux. There are three ways
for the operator to send commands to \textbf{manip}:
\begin{itemize}
\item From the \textbf{calibration} console in the DCR.
\item From the \textbf{manmon} program running on a number of Linux
  boxes.
\end{itemize}
When changing source or adjusting equipment in the DCR, the console is
usually used. Otherwise the calibration operator usually interacts
with the manipulator through \textbf{manmon}. The manipulator can be
operated both from underground or from surface without any personnel
in the lab. There is also a monitoring program running on
\textbf{buffer1.sp.snolab.ca}. This program is called
\textbf{manip\_logger} which generates a json document containing the
current manipulator status and uploads the result to a couch database
on the \textbf{snopl.us} server. The resulting files may be viewed
from a web browser or downloaded directly.

\section{Controlling and Monitoring the Manipulator from manmon}
Nominal operatiorn of the mainipulator is through \textbf{manmon}
which is a Tcl/Tk program which runs on various Linux workstations on
site. Specifically the workstations

\paragraph{monug1} The workstation in the underground control room.
\paragraph{monag1} The workstation in the surface control room.
\paragraph{monlu1} The workstation in the Laurentian control room.

\textbf{manmon} provides a GUI interface to display the manipulator
data and to allow the Calibration Operator to move sources or operate
the calibration laser.

\subsection{Starting manmon}
\begin{enumerate}
\item Log on to \textbf{monug1} as \textbf{snoperator}
\item Go to the current manmon directory.
\begin{verbatim}
cd ~/manmon_3.00
\end{verbatim}
\item Start the \textbf{manmon} program by typing
\begin{verbatim}
manmon
\end{verbatim}
A GUI user interface will pop up Fig.~\ref{fig:manmonGUI}.
\item The GUI connects to the calibration computer automatically, but
  in case it does not, the connection may be forced by clicking on the
  \textbf{connect} button. A window will pop up (see
  Fig.\ref{fig:connect}) 
\end{enumerate}
The procedure is identical in the other control rooms.


\section{Controlling the Manipulator from the Console}
On the calibration computer, change to the directory
\begin{verbatim}
~/AVR32
\end{verbatim}
and run the program
\begin{verbatim}
./manip
\end{verbatim}
The basic commands are:
\paragraph{help}
\paragraph{list} lists all objects
\paragraph{logout} disconnects the tcp/ip connection
\paragraph{quit} exists program
Commands for the \verb+prototyp+ object (typically a source object
like \verb+laserball+, \verb+ambe+, \verb+li8+, etc.)
