

\section{SNO+ Geometry and the Calibration System}

When using the manipulator the user is positioning the carriage pivot
in a coordinate system fixed to the AV. In SNO+ the acrylic vessel is
a free floating entity. Measurements by the detector are collected by
the PMT support structure which is fixed relative to the
cavity. Neither the PSUP nor the AV are expected to be possess a
mutual sensor, so the translation between the AV coordinate system and
the PSUP coordinate system must be independently measured. The
dimensions of the AV are given in Table~\ref{tab:avdims}.

\begin{table}
  \begin{center}
    \begin{tabular}{llll}
      \hline
      measurement & dim(in) & dim(cm) & source \\
      \hline
      height of DCR floor & 100'=1200'' & 3048.00 & definition \\
      height of global origin & 56' 7 1/2'' = 679.5'' & 1725.93 &
      design \\
      d(global origin to DCR floor) & 520.5'' & 1322.07 & calc. \\
      \hline
    \end{tabular}
  \end{center}
  \caption{Global Coordinate System}
  \label{tab:avdims}
\end{table}

\begin{table}
  \begin{center}
    \begin{tabular}{llll}
      \hline
      Measurement & Dim(in) & Dim(cm) & Source \\
      \hline
      Height of 10'' gate valve from UI & 13 & & \\
      Height of 6'' gate valve from UI & 8.25 & & \\
      \hline
      Height of Upper UI & 18.75 & & from Drawings\\
      Lower UI top flange width & 0.87 & & from Drawings\\
      Lower UI wall height & 28 & & from Drawings \\
      Lower UI bottom flange width & 0.75 & & measured \\
      UI height from top of lower UI bottom flange to top surface & 47.5625 & & measured \\
      Gap between AV and UI & 0.5 & & measured \\
      Height of UI top surface above AV center & 567.9725 & & calc. \\
      \hline
      URM1 source tube height to crosshairs & 41.85 & 106.3 & measured \\
      URM1 crosshair height & 609.8228 & 1548.95 & calc. \\
      URM2 source tube height to crosshairs & 40.812 & 103.66 & measured \\ 
      URM2 crosshair height & 609.9725 & 1549.33 & calc. \\
      URM3 source tube height to crosshairs & 43.819 & 111.3 & measured \\ 
      URM3 crosshair height & 607.5725 & 1542.23 & calc. \\
      \hline
    \end{tabular}
  \end{center}
  \caption{}
  \label{tab:UIheights}
\end{table}

\begin{table}
  \begin{center}
    \begin{tabular}{llll}
      \hline
      Measurement & AV height(in) & AV height(cm) & source \\
      \hline
      Laser level & & \\
      
      \hline
    \end{tabular}
  \end{center}
  \caption{Dimensions of the SNO+ detector.}
  \label{tab:avpsupmeas}
\end{table}

\begin{table}
  \begin{center}
    \begin{tabular}{llll}
      \hline
      measurement & dim (in)& dim (cm)& from \\
      Average Vessel Inner Radius & 236.38$\pm$0.23'' &
      600.41$\pm$0.58 & measurement\\ 
      Top of Chimney to AV bottom & 742.59$\pm$0.05'' &
      1886.18$\pm$0.13 & measurement\\ 
      Top of Chimney to AV centre & 506.16$\pm$0.12'' & 1285.65$\pm$0.30 &
      measured? \\
      Neck Ring gasket & 1/8'' & 0.3175 & measured \\
      Neck Ring plate & 3/8'' & 0.9525 & measured \\
      AV Centre to AV top plate & 506.66'' & 1286.92 & calculated \\
      DCR floor to AV top plate & 12.4375'' & 31.59 & measured/calculated\\
      \hline
    \end{tabular}
  \end{center}
  \caption{Dimensions of the SNO+ detector.}
  \label{tab:avpsupmeas}
\end{table}



There are multiple methods for the measurement of the AV with respect
to the AV. The first is a direct measurement of the top of the UI
relative to the PSUP anchors using a laser level. This requires
opening part of the light seal for the cavity so it is not something
that can be repeated with a high regularity. Measurements using
reflections of light from TELLIE has provided a very good measurement
of the position of the AV with respect to the PSUP. This analysis can
be repeated with the further TELLIE runs. Further measurements can be
completed from the analysis of changes in the tensions on the hold
down ropes although the initial lengths and positions are not well
understood. The results of these measurements are given in
Table~\ref{tab:avpsupmeas}.

The relative positions of the guide tube gate valves have been
measured with respect to the top of the AV. In the water phase the
gate valves from SNO were used due to the low tolerances of the gate
valves aquired for SNO+. These positions then give the total distance
of the source from the center of the detector given the height of the
source tubes. The measurements were conducted using a laser level and
a measuring tape and compiled in Table~\ref{tab:guidetubeheight}.

\begin{table}
  \begin{center}
    \begin{tabular}{cccccc}
      \hline
      Tube & X (cm)& Y (cm)& Enter PSUP & Z Leaves PSUP & Z Touches
      AV(cm)\\
      \hline
      1&361.00&193.04& 734.36& &446.91 \\
      2&-112.08& 104.14& 825.47& &586.44 \\
      3&-361.00& 193.04& 733.51& &446.91 \\
      4&-586.11& 207.96& 564.64& -564.64 & \\
      5&-586.11&-252.41& 546.23& -546.23 & \\
      6& 118.11&-119.38& 823.80& & 582.32\\
      \hline
    \end{tabular}
  \end{center}
  \caption{Calibration Guide Tube Locations.}
  \label{tab:guidtubeheight}
\end{table}

\section{Acrylic Vessel}
The thermal expansion coefficient for the acrylic is
\begin{math}
  6\times 10^{-5}\textrm{C}^{-1}
\end{math}
The design specs for the AV give the distance from the top of the neck
to the centre of the vessel at 23 C
\begin{verbatim}
42' 2 3/8''
\end{verbatim}
which is 506.375~cm and the nominal outside radius is
\begin{verbatim}
236.6''
\end{verbatim}
which is 600.964~cm with a nominal thickness of 2.15'' (5.461~cm).

This can be compared to the results found in SNO-STR-98-003 (R. Komar)
for actual measurements of the AV. Using the Komar measurements, the
nominal dimensions of the AV are given in Table~\ref{tab:}. On top of the AV
neck flange is a gasket (1/8'') and a stainless steel top plate
(3/8''). This gives a distance from the centre of the AV to the top
plate of,
\begin{math}
  506.16 + 1/8 + 3/8 = 506.66\textrm{in} = 1286.92\textrm{cm}
\end{math}

Recent measurements of the 

\section{Calibration Guide Tubes}

The locations of the calibration guide tubes in the Deck Clean Room
are shown in Fig.~\ref{fig:deckplan} and Fig.~\ref{fig:gtelev}. The
heights of the gate valves for the guide tubes as well as the heights
of the mounted source tubes at the cross hairs is given in
Table~\ref{tab:gtheights}.

\begin{figure}

  \caption{Deck plan of the DCR showing the positions of the guide
    tubes}
  \label{fig:deckplan}
\end{figure}

\begin{figure}

  \caption{Elevations of the calibration guide tubes}
  \label{fig:gtelev}
\end{figure}
